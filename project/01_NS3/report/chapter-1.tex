\chapter{Introduction}

In recent decades, wireless communications have gained a lot of attentions in the area of communications, especially for developing mobile telephony~\cite{athanasopoulos2006evaluation}. WLANs, nowadays, can support a variety of industrial and home automation applications such as Internet of Things (IoT). WLANs can be divided into two main categories based on the topology architecture, namely: infrastructure WLANs Access Point (AP) and ad-hoc WLANs. In AP-infrastructure WLANs, at least one AP acts as a server in which all kind of communication pass through this AP. In ad-hoc WLANs, each node can communicate autonomously point-to-point without having to communicate through AP. Most of the network use infrastructure mode due to its simplicity and scalability~\cite{crow1997ieee}.

As the WLANs are widely used in many areas, cross-vendor industries have emerged. Thus, a standard is needed to ensure compatibility among WLAN devices. One of the specification is  
IEEE 802.11. The IEEE 802.11 is a set of media access control (MAC) and physical layer (PHY) specifications for implementing wireless local area network (WLAN) computer communication in the 900 MHz and 2.4, 3.6, 5, and 60 GHz frequency bands. This specification is created and maintained by Institute of Electrical and Electronics Engineers (IEEE) LAN/MAN Standards Committee (IEEE 802). The IEEE 802.11 was released in 1997 and since then there are several modifications, for instance, IEEE 802.11b and IEEE 802.11g. IEEE 802.11b is the most popular standard of all 802.11 families. IEEE 802.11g is just an extension of IEE 802.11b, supporting higher data rates.

The IEEE 802.11b standard employs Direct Sequence Spread Spectrum (DSSS) with Complementary Code Keying (CCK) as the modulation scheme and operates on the ISM 2.4 GHz frequency band. It has 22 MHz bandwidth and provides four different data rates: 1, 2, 5.5, and 11 MBit/s. IEEE 802.11b has an indoor range around 35 m and outdoor range around 140 m.

The IEEE 802.11b uses CSMA/CA technique for transmitting a packet and avoiding collision. It works by first sensing the presence of any signal in the channel, waiting for the channel to be empty before transmitting a packet. After successfully transmitted the packet, the transmitter will receive an acknowledgement from the receiver. However, if there is a collision, there will be no acknowledgement received and the transmitter assume the packet was lost. Then, the transmitter will do the same process, sensing the channel and retransmitting the packet whenever the channel is empty after a certain amount of time, which is called backoff time. Thus, it makes sense that the more users in a particular area, the greater the probability of occurrence of collisions, which may lead to the decreasing throughput.

As the IEEE 802.11b uses DSSS in which one frequency channel can be occupied by many users, power and interference become two dominant factors affecting the throughput~\cite{kemerlis2006throughput}. For instance, a user that is located far away from an AP will get a lower signal level and probably get a lower throughput than a user that is located near AP. However, IEEE 802.11b provides advanced scheduling and traffic control~\cite{bianchi2000performance}. Therefore, in this project, the density of users, location, and mobility that predominantly affect the througput will be investigated.