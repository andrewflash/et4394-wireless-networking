\chapter{Conclusion}

Energy detection is indeed straightforward and easy to implement for detecting the presence of a signal without demodulating the signal itself, but it is hard to classify such kind of signals. By simply adding the ability of the detector to identify the signal based on bandwidth, we can improve the performance of the detector and reduce the probability of false alarm. Moreover, we can perform cooperative sensing by using multiple receivers located in different places in the area of interest. The results then can be combined and analyzed to get more precise results. Measurement sampling should also be increased to get better results when conducting the analysis of receiver characteristic (ROC).

Threshold value essentially affect the performance of the detector. Threshold value causes a trade-off between probability of detection and probability of false alarm. Therefore, threshold value should be selected carefully to get the optimal performance of the detector. By using ROC analysis, the optimal threshold value can be obtained.